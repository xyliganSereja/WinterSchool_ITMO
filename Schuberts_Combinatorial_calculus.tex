\documentclass{article}
\usepackage{bbold}
\usepackage{amsfonts}
\usepackage{amsmath}
\usepackage{amssymb}
\usepackage{color}
\setlength{\columnseprule}{1pt}
\usepackage{cmap}
\usepackage[utf8]{inputenc}
\usepackage[T2A]{fontenc}
\usepackage[english, russian]{babel}
\usepackage{graphicx}
\usepackage{hyperref}
\usepackage{mathdots}
\usepackage{xfrac}


\def\columnseprulecolor{\color{black}}

\graphicspath{ {./resources/} }


\usepackage{listings}
\usepackage{xcolor}
\definecolor{codegreen}{rgb}{0,0.6,0}
\definecolor{codegray}{rgb}{0.5,0.5,0.5}
\definecolor{codepurple}{rgb}{0.58,0,0.82}
\definecolor{backcolour}{rgb}{0.95,0.95,0.92}
\lstdefinestyle{mystyle}{
    backgroundcolor=\color{backcolour},   
    commentstyle=\color{codegreen},
    keywordstyle=\color{magenta},
    numberstyle=\tiny\color{codegray},
    stringstyle=\color{codepurple},
    basicstyle=\ttfamily\footnotesize,
    breakatwhitespace=false,         
    breaklines=true,                 
    captionpos=b,                    
    keepspaces=true,                 
    numbers=left,                    
    numbersep=5pt,                  
    showspaces=false,                
    showstringspaces=false,
    showtabs=false,                  
    tabsize=2
}

\lstset{extendedchars=\true}
\lstset{style=mystyle}

\newcommand\0{\mathbb{0}}
\newcommand{\eps}{\varepsilon}
\newcommand\overdot{\overset{\bullet}}
\DeclareMathOperator{\sign}{sign}
\DeclareMathOperator{\re}{Re}
\DeclareMathOperator{\im}{Im}
\DeclareMathOperator{\Arg}{Arg}
\DeclareMathOperator{\const}{const}
\DeclareMathOperator{\rg}{rg}
\DeclareMathOperator{\Span}{span}
\DeclareMathOperator{\alt}{alt}
\DeclareMathOperator{\Sim}{sim}
\DeclareMathOperator{\inv}{inv}
\DeclareMathOperator{\dist}{dist}
\DeclareMathOperator{\supp}{supp}
\DeclareMathOperator{\Var}{Var}
\DeclareMathOperator{\Cov}{Cov}
\DeclareMathOperator{\argmax}{argmax}
\DeclareMathOperator{\argmin}{argmin}
\DeclareMathOperator{\med}{med}
\newcommand\1{\mathbb{1}}
\newcommand\ul{\underline}
\newcommand{\ppart}[2]{\frac{\partial #1}{\partial #2}}
\renewcommand{\bf}{\textbf}
\renewcommand{\it}{\textit}
\newcommand\vect{\overrightarrow}
\newcommand{\nm}{\operatorname}
\DeclareMathOperator{\df}{d}
\DeclareMathOperator{\tr}{tr}
\newcommand{\bb}{\mathbb}
\newcommand{\lan}{\langle}
\newcommand{\ran}{\rangle}
\newcommand{\an}[2]{\lan #1, #2 \ran}
\newcommand{\fall}{\forall\,}
\newcommand{\ex}{\exists\,}
\newcommand{\lto}{\leftarrow}
\newcommand{\xlto}{\xleftarrow}
\newcommand{\rto}{\rightarrow}
\newcommand{\xrto}{\xrightarrow}
\newcommand{\uto}{\uparrow}
\newcommand{\dto}{\downarrow}
\newcommand{\lrto}{\leftrightarrow}
\newcommand{\llto}{\leftleftarrows}
\newcommand{\rrto}{\rightrightarrows}
\newcommand{\Lto}{\Leftarrow}
\newcommand{\Rto}{\Rightarrow}
\newcommand{\Uto}{\Uparrow}
\newcommand{\Dto}{\Downarrow}
\newcommand{\LRto}{\Leftrightarrow}
\newcommand{\Rset}{\bb{R}}
\newcommand{\Rex}{\overline{\bb{R}}}
\newcommand{\Cset}{\bb{C}}
\newcommand{\Nset}{\bb{N}}
\newcommand{\Qset}{\bb{Q}}
\newcommand{\Zset}{\bb{Z}}
\newcommand{\Bset}{\bb{B}}
\renewcommand{\ker}{\nm{Ker}}
\renewcommand{\span}{\nm{span}}
\newcommand{\Def}{\nm{def}}
\newcommand{\mc}{\mathcal}
\newcommand{\mcA}{\mc{A}}
\newcommand{\mcB}{\mc{B}}
\newcommand{\mcC}{\mc{C}}
\newcommand{\mcD}{\mc{D}}
\newcommand{\mcJ}{\mc{J}}
\newcommand{\mcT}{\mc{T}}
\newcommand{\us}{\underset}
\newcommand{\os}{\overset}
\newcommand{\ol}{\overline}
\newcommand{\ot}{\widetilde}
\newcommand{\vl}{\Biggr|}
\newcommand{\ub}[2]{\underbrace{#2}_{#1}}
\newcommand{\ob}[2]{\overbrace{#2}^{#1}}
\newcommand{\pat}{\partial}

\def\letus{%
    \mathord{\setbox0=\hbox{$\exists$}%
             \hbox{\kern 0.125\wd0%
                   \vbox to \ht0{%
                      \hrule width 0.75\wd0%
                      \vfill%
                      \hrule width 0.75\wd0}%
                   \vrule height \ht0%
                   \kern 0.125\wd0}%
           }%
}
\DeclareMathOperator*\dlim{\underline{lim}}
\DeclareMathOperator*\ulim{\overline{lim}}

\everymath{\displaystyle}

% Grath
\usepackage{tikz}
\usetikzlibrary{positioning}
\usetikzlibrary{decorations.pathmorphing}
\tikzset{snake/.style={decorate, decoration=snake}}
\tikzset{node/.style={circle, draw=black!60, fill=white!5, very thick, minimum size=7mm}}
\newcommand{\norm}[1]{\left\lVert#1\right\rVert}
\title{\hugeКомбинаторное исчисление Шуберта}
\author{Матвеев Сергей M3338}
\date{Зимняя школа по математике}
\begin{document}

\maketitle
\section{Геометрические аспекты}
$Gr(k, n) = \{V \subset \mathbb{C}^n\text{ }dim(B) = k\}$ - грасманиан\\
$\{0\} = F_0 \subset F_1 \subset \dots \subset F_n = \mathbb{C}^n$\\
$dim F_i = i$\\
$F_.$ - флаг\\
Можно просто думать про базис, где $E_i = span{e_1, \dots, e_i}$\\
Пусть есть два флага: $E_., F_.$\\
$max(0, i + j - n) \leq dim(E_i \cap F_j) \leq min(i, j)$\\
Для одинаковых флагов $dim(E_i \cap E_j) = min(i, j)$\\
$E_i = span(e_1, \dots, e_i)$\\
$F_j = span(e_n, e_{n - 1}, \dots, e_{n - j + 1})$ - тоже флаг\\
$dim(E_i \cap F_j) = max(0, i + j - n)$\\
Эти флаги называются флагами общего положения\\
$V \in Gr(k, m); e_1, \dots, e_n$ - базис $\mathbb{C}^n$\\
$M_{k \times n}, rank(M) = k$\\
$V(M)$ - эту матрица будет иметь вид, где в каждой строке есть единичка и после нее сколько-то нулей\\
Где в строке i будет $k - i + \lambda_i$ - ноль\\
$\lambda_1 \geq \lambda_2 \geq \dots \geq \lambda_k$ - это разбиение / диаграмма Юнга\\
$F_.$ - флаг\\
\bf{Клетка Шуберта} это $\Omega_\lambda(F_.) = \{V \in Gr(k, n) | pos(V, F_.) = \lambda \}$\\
$Gr(k, n) = $ разбиение $\lambda$ - диаграмма в $k \times n - k$ $\Omega_\lambda(F_.)$\\
$Gr(2, 4) = \Omega_0(F_.) \cup \Omega_1(F_.) \cup \Omega_2(F_.) \cup \Omega_{2^T}(F_.) \cup \Omega_3(F_.) \cup \Omega_4(F_.)$\\
$|\lambda| = \lambda_1 + \lambda_2 + \dots, \lambda_k$ - размер диаграммы\\
\bf{Утверждение} $dim\Omega_\lambda(F_.) = k(n - k) - |\lambda|$\\
Или $codim\Omega_\lambda(F_.) = |\lambda|$\\
Количество нулей - $|\lambda| + k(k - 1)$\\
Количество единиц - $k$\\
Тогда у нас остается $kn - (|\lambda| + (k - 1)k) - k$ - позиций где мы можем расставить чиселки\\
\bf{Цикл Шуберта} - $X_\lambda(F_.) = \overline{\Omega_\lambda(F_.)} = \bot\bot_{\mu \geq \lambda} \Omega_\mu(F_.)$\\
$\lambda' -$ это $\lambda$ с дополнительным квадратиком, то есть у нас единичка переедет на одну ячейку вправо\\
$Gr(2, 4), \Lambda_1, \dots, \Lambda_n$, n = 4 - прямые в $\mathbb{C}^4$\\
$E_2^i$ - плоскость в $\mathbb{C}^4$ которая содержит $\Lambda_i$ и $\{0\}$\\
$E_.^i$ - флаг $E_1^i$ и $E_3^i$ выбр (общего положения)\\
$X_1(E_.^1) \cap X_1(E_.^2) \cap X_1(E_.^3) \cap X_1(E_.^4)$ - это какое-то число элементов в $Gr(2, 4)$\\
$X_1(E_.^i)$ - это двухмерные плоскости которые пересекают $\lambda_i$\\
Получается $X_1(E_.^1) \cap X_1(E_.^2) \cap X_1(E_.^3) \cap X_1(E_.^4)$ - это количество прямых, которые перескают $\Lambda_1, \Lambda_2, \Lambda_3, \Lambda_4$\\
$\lambda^{(1)}, \dots, \lambda^{(m)}$ - m диаграмм Юнга\\
$|\lambda^{(1)}| + \dots + |\lambda^{(m)}|$\\
$E_.^{(1)}, \dots, E_.^{(m)}$ - флаги общего положения\\
Тогда возникает вопрос (Исчисление Шуберта), сколько точек в $(X_{\lambda^{(1)}}(E_.^1) \cap \dots \cap X_{\lambda^{(m)}}(E_.^m))$\\
$\lambda, \mu, \eta$ - три диаграммы в $k \times (n - k): |\eta| = |\lambda| + |\mu|$\\
\bf{Коэффициент Литтлвуда-Ричардсона}\\
$C_{\lambda, \mu}^\eta = \#(X_\lambda() \cap X_\mu() \cap X_\eta())$
\end{document}