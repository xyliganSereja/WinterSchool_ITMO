\documentclass{article}
\usepackage{bbold}
\usepackage{amsfonts}
\usepackage{amsmath}
\usepackage{amssymb}
\usepackage{color}
\setlength{\columnseprule}{1pt}
\usepackage{cmap}
\usepackage[utf8]{inputenc}
\usepackage[T2A]{fontenc}
\usepackage[english, russian]{babel}
\usepackage{graphicx}
\usepackage{hyperref}
\usepackage{mathdots}
\usepackage{xfrac}


\def\columnseprulecolor{\color{black}}

\graphicspath{ {./resources/} }


\usepackage{listings}
\usepackage{xcolor}
\definecolor{codegreen}{rgb}{0,0.6,0}
\definecolor{codegray}{rgb}{0.5,0.5,0.5}
\definecolor{codepurple}{rgb}{0.58,0,0.82}
\definecolor{backcolour}{rgb}{0.95,0.95,0.92}
\lstdefinestyle{mystyle}{
    backgroundcolor=\color{backcolour},   
    commentstyle=\color{codegreen},
    keywordstyle=\color{magenta},
    numberstyle=\tiny\color{codegray},
    stringstyle=\color{codepurple},
    basicstyle=\ttfamily\footnotesize,
    breakatwhitespace=false,         
    breaklines=true,                 
    captionpos=b,                    
    keepspaces=true,                 
    numbers=left,                    
    numbersep=5pt,                  
    showspaces=false,                
    showstringspaces=false,
    showtabs=false,                  
    tabsize=2
}

\lstset{extendedchars=\true}
\lstset{style=mystyle}

\newcommand\0{\mathbb{0}}
\newcommand{\eps}{\varepsilon}
\newcommand\overdot{\overset{\bullet}}
\DeclareMathOperator{\sign}{sign}
\DeclareMathOperator{\re}{Re}
\DeclareMathOperator{\im}{Im}
\DeclareMathOperator{\Arg}{Arg}
\DeclareMathOperator{\const}{const}
\DeclareMathOperator{\rg}{rg}
\DeclareMathOperator{\Span}{span}
\DeclareMathOperator{\alt}{alt}
\DeclareMathOperator{\Sim}{sim}
\DeclareMathOperator{\inv}{inv}
\DeclareMathOperator{\dist}{dist}
\DeclareMathOperator{\supp}{supp}
\DeclareMathOperator{\Var}{Var}
\DeclareMathOperator{\Cov}{Cov}
\DeclareMathOperator{\argmax}{argmax}
\DeclareMathOperator{\argmin}{argmin}
\DeclareMathOperator{\med}{med}
\newcommand\1{\mathbb{1}}
\newcommand\ul{\underline}
\newcommand{\ppart}[2]{\frac{\partial #1}{\partial #2}}
\renewcommand{\bf}{\textbf}
\renewcommand{\it}{\textit}
\newcommand\vect{\overrightarrow}
\newcommand{\nm}{\operatorname}
\DeclareMathOperator{\df}{d}
\DeclareMathOperator{\tr}{tr}
\newcommand{\bb}{\mathbb}
\newcommand{\lan}{\langle}
\newcommand{\ran}{\rangle}
\newcommand{\an}[2]{\lan #1, #2 \ran}
\newcommand{\fall}{\forall\,}
\newcommand{\ex}{\exists\,}
\newcommand{\lto}{\leftarrow}
\newcommand{\xlto}{\xleftarrow}
\newcommand{\rto}{\rightarrow}
\newcommand{\xrto}{\xrightarrow}
\newcommand{\uto}{\uparrow}
\newcommand{\dto}{\downarrow}
\newcommand{\lrto}{\leftrightarrow}
\newcommand{\llto}{\leftleftarrows}
\newcommand{\rrto}{\rightrightarrows}
\newcommand{\Lto}{\Leftarrow}
\newcommand{\Rto}{\Rightarrow}
\newcommand{\Uto}{\Uparrow}
\newcommand{\Dto}{\Downarrow}
\newcommand{\LRto}{\Leftrightarrow}
\newcommand{\Rset}{\bb{R}}
\newcommand{\Rex}{\overline{\bb{R}}}
\newcommand{\Cset}{\bb{C}}
\newcommand{\Nset}{\bb{N}}
\newcommand{\Qset}{\bb{Q}}
\newcommand{\Zset}{\bb{Z}}
\newcommand{\Bset}{\bb{B}}
\renewcommand{\ker}{\nm{Ker}}
\renewcommand{\span}{\nm{span}}
\newcommand{\Def}{\nm{def}}
\newcommand{\mc}{\mathcal}
\newcommand{\mcA}{\mc{A}}
\newcommand{\mcB}{\mc{B}}
\newcommand{\mcC}{\mc{C}}
\newcommand{\mcD}{\mc{D}}
\newcommand{\mcJ}{\mc{J}}
\newcommand{\mcT}{\mc{T}}
\newcommand{\us}{\underset}
\newcommand{\os}{\overset}
\newcommand{\ol}{\overline}
\newcommand{\ot}{\widetilde}
\newcommand{\vl}{\Biggr|}
\newcommand{\ub}[2]{\underbrace{#2}_{#1}}
\newcommand{\ob}[2]{\overbrace{#2}^{#1}}
\newcommand{\pat}{\partial}

\def\letus{%
    \mathord{\setbox0=\hbox{$\exists$}%
             \hbox{\kern 0.125\wd0%
                   \vbox to \ht0{%
                      \hrule width 0.75\wd0%
                      \vfill%
                      \hrule width 0.75\wd0}%
                   \vrule height \ht0%
                   \kern 0.125\wd0}%
           }%
}
\DeclareMathOperator*\dlim{\underline{lim}}
\DeclareMathOperator*\ulim{\overline{lim}}

\everymath{\displaystyle}

% Grath
\usepackage{tikz}
\usetikzlibrary{positioning}
\usetikzlibrary{decorations.pathmorphing}
\tikzset{snake/.style={decorate, decoration=snake}}
\tikzset{node/.style={circle, draw=black!60, fill=white!5, very thick, minimum size=7mm}}
\newcommand{\norm}[1]{\left\lVert#1\right\rVert}
\title{\hugeСуб-финслерова геометрия и выпуклая тригонометрия}
\author{Матвеев Сергей M3338}
\date{Зимняя школа по математике}
\begin{document}

\maketitle
\section{Постановка задачи}
Пусть дана плоскость $\mathbb{R}^2$ и фигура $S_0$, $l \to min, S = S_0$\\
Если мерить по евклиду, то ответ это просто круг, но давайте мерить расстояния по другому, то ответ может быть другим, например в $L^1$ это квадрат\\
Давайте научимся решать эту задачу, методами вариационного исчисления\\
У нас есть глобальное (неудобное ограничение), пусть мы нарисовали часть кривой, тогда ее длину мы можем посчитать сразу, а вот площадь нет, давайте действовать по другому\\
$l = (x(t), y(t)), t \in [0, 1]$\\
$\mathbb{R}^2$\\
$x(0) = x(1)$\\
$y(0) = y(1)$\\
По формуле Грина $S = \frac{1}{2}\displaystyle\int_0^1(xy' - x'y) dt = S_0$\\
$l = \displaystyle\int_0^1 \sqrt{x'^2 + y'^2} dt \to min$\\
$z(t) = \frac{1}{2}\displaystyle\int_0^t(x(s)y'(s) - x'(s)y(s)) ds \LRto z(0) = 0, z'(t) = \frac{1}{2}(x(t)y'(t) - x'(t)y(t))$\\
$\mathbb{R}^3 (x(t), y(t), z(t))$\\
$x(0) = x(1)$\\
$y(0) = y(1)$\\
$z(0) = z(1) = S_0$\\
$l = \displaystyle\int_0^1 \sqrt{x'^2 + y'^2}dt \to min$\\
$Z' = \frac{1}{2}(xy' - x'y)$ - неголономная связь\\
Мы рассматриваем кривые удолетворяющие условию: $a(x, y, z)x' + b(x, y, z)y' + c(x, y, z)z' = 0 (*)$\\
Задача, когда $\forall A, B \in \mathbb{R}^3, \exists$ кривая (*) соединяющая и A и B\\
Суб-Риманова геометрия - у нас дано многообращие и в кадой точке у нас есть касательное подпространство разрешенных направлений, а Финслерова означает, что мы мерием расстояние как-то по другому\\
Когда мы ищем минимум, у нас может быть их несколько (действительно, давайте просто придумаем дурацкие параметризации), давайте тогда перейдем к такой задаче\\
$\displaystyle\int_0^1(x'^2 + y'^2)dt \to min$\\
Коши-Буняковского:\\
$f = \sqrt{x'^2 + y'^2}, g = 1$
$\displaystyle\int_0^1 fg dt \leq (\displaystyle\int_0^1 f^2 dt)^\frac{1}{2}(\displaystyle\int_0^1 g^2)^\frac{1}{2}$\\
$l \leq \sqrt{E}, E \to min$\\
Кривая называется натурально параметризованной, если скорость в каждой точке является константой\\
У нас получается задача:\\
$\displaystyle\int_0^1 (x'^2 + y'^2)dt \to min$\\
$\frac{1}{2}\displaystyle\int_0^1 (xy' - x'y)dt = S_0$\\
Давайте напишем лаграндиан:\\
$\mathcal{L} = \lambda_0 \displaystyle\int_0^1 (x'^2 + y'^2)dt + \frac{\lambda_1}{2}\displaystyle\int_0^1 (xy' - x'y)dt - \lambda_1 S_0$\\
$\mathcal{L}' = 0$\\
Теперь начнем писать уравнение Эйлера-Лагранжа:\\
$\frac{d}{dt}L_{x'}' = L_x'$\\
$\frac{d}{dt}L_{y'}' = L_y'$\\
$L = \lambda_0(x'^2 + y'^2) + \frac{\lambda_1}{2}(xy' - x'y)$\\
$\frac{d}{dt}(2\lambda_0x' - \frac{\lambda_1}{2}y) = \frac{\lambda_1}{2}y'$\\
$\frac{d}{dt}(2\lambda_0 y' + \frac{\lambda_1}{2}x) = -\frac{\lambda_1}{2}x'$\\
Тогда если решения существуют то они являются решениями системы:\\
$\begin{cases}
    2\lambda_0x'' = \lambda_1 y'\\
    2\lambda_0 y'' = -\lambda_1x'
\end{cases}$
\begin{enumerate}
    \item Аномальный случай: $\lambda_0 = 0, \lambda_1 \neq 0$
    $\begin{cases}
        x' = 0\\
        y' = 0
    \end{cases} \Rto x(t) = const, y(t) = const$ - скучно
    \item $\lambda_0 \neq 0 \Rto$ Б.о.о $\lambda_1 = \frac{1}{2}$\\
    $\begin{cases}
        x'' = \lambda_1y'\\
        y'' = -\lambda_1x'
    \end{cases}$\\
    $\begin{cases}
        x' = R \cos(\theta)\\
        y' = R \sin(\theta)
    \end{cases}$\\
    $\begin{cases}
        R = \sqrt{x'^2 + y'^2}\\
        R' = \frac{2x'x'' + 2y'y''}{2R}\\
        \theta = \arctan(\frac{y'}{x'})
    \end{cases}$\\
    $\begin{cases}
        R' = x'' \cos(\theta) + y'' \sin(\theta)\\
        \theta' = -\frac{1}{R}(y''\cos(\theta) - x''\sin(\theta))
    \end{cases}$\\
    $R' = \lambda_1R \sin(\theta)\cos(\theta) + (-\lambda_1)R\cos(\theta)\sin(\theta) = 0$\\
    $\theta' = \frac{-1}{R}(-\lambda_1 R \cos^2(\theta) - \lambda_1 R \sin^2(\theta))$\\
    $\begin{cases}
        R' = 0\\
        \theta' = \lambda_1
    \end{cases}$\\
    $\theta = at = b$ и просто подставим\\
\end{enumerate}
Пусть $\norm{\cdot}$ - какая-то норма на $\mathbb{R}^2$\\
$\begin{cases}
    l \to min\\
    S = S_0
\end{cases}$\\
$l = \displaystyle\int_0^1 \norm{(x'(t), y'(t))}dt$\\
$S = \frac{1}{2}\displaystyle\int_0^1 (xy' - x'y)dt$\\
$E = \displaystyle\int_0^1 \norm{(x', y')}^2 dt \to min$\\
$S = \frac{1}{2}\displaystyle\int_0^1 (xy' - x'y)dt = S_0$\\
$\mathcal{L} = \lambda_0 \displaystyle\int_0^1 \norm{(x', y')}^2dt + \frac{\lambda_1}{2}\displaystyle\int_0^1(xy' - x'y)dt - \lambda_1 S_0$\\
$L = \lambda_0 \norm{(x', y')}^2 + \frac{\lambda_1}{2}(xy' - x'y)$\\
$\norm{x', y'} = R$\\
Теперь давайте посмотрим на общий случай:\\
$\begin{cases}
    \frac{d}{dt}(2\lambda_0R R'_{x'} = \lambda_1y')\\
    \frac{d}{dt}(2\lambda_0R R'_{y'} = -\lambda_1x')\\
    R = \norm{(x', y')}
\end{cases}$\\
$\Omega = \{(x, y): \norm{(x, y)} \leq 1\} \subset \mathbb{R}^2$\\
$\norm{(x, y)}_1 = |x| = |y|$\\
$\norm{(x, y)}_\infty = max(|x|, |y|)$\\
$\norm{(x, y)}_e = \sqrt{\frac{x^2}{a^2} + \frac{y^2}{b^2}}$\\
$\alpha > 1$\\
$\norm{(x, y)}_\alpha = (|x|^\alpha + |y|^\alpha)^{\frac{1}{\alpha}}$\\
\section{Выпуклая тригонометрия}
Пусть $\Omega \subset \mathbb{R}^2$ - выпуклое компактное множество, $o \in int\Omega$\\
$\forall \theta \in [0, 2S(\Omega))$\\
$A_\theta = (\cos_\Omega\theta, \sin_\Omega\theta)$ если $\theta \notin [0, 2S(\Omega)) \Rto$ периодическая с периодом $2S(\Omega)$\\
$\Omega \subset \mathbb{R}^2 = \{(x, y)\}$\\
$\Omega^\circ \subset \mathbb{R}^{2*} = \{(p, q)\}$\\
\bf{Определение. Поляры.}\\
$\Omega^\circ = \{(p, q): \forall (x, y) \in \Omega, px + qy \leq 1\}$\\
\bf{Теорема. О биполяре}\\
$\Omega -$ выпуклая, замкнутая и $o \in \Omega,$ то $\Omega^{\circ \circ} = \Omega$\\
\bf{Доказательство}\\
1. $\Omega \subset \Omega^{\circ \circ}$\\
$\forall (x, y) \in \Omega$ $\forall (p, q) \in \Omega^\circ$\\
$px + qy \leq 1$\\
$\Omega^{\circ \circ} = \{(\widetilde{x}, \widetilde{y}: \forall (p, q) \in \Omega\text{  }\widetilde{x}p + \widetilde{y}q \leq 1)\}$\\
2. $\Omega^{\circ \circ} \backslash \Omega = \emptyset \Rto (p, q) \in \Omega^\circ$\\
$(\widetilde{x}, \widetilde{y}) \notin \Omega^{\circ \circ}$\\
\\
$B_\psi = (\cos_{\Omega^\circ} \psi, \sin_{\Omega^\circ} \psi)$\\
\bf{Определение}\\
$\theta \lrto \theta^\circ$\\
Касательная в $A_\theta$ к $\Omega$ опр ед $B_{\theta^\circ}$ из $\Omega^\circ$\\
\bf{Теорема Пифагора}\\
$\cos_\Omega\theta \cos_{\Omega^\circ} \psi + \sin_\Omega\theta \sin_{\Omega^\circ} \psi \leq 1$\\
Но $\cos_\Omega\theta \cos_{\Omega^\circ} \theta^\circ + \sin_\Omega\theta \sin_{\Omega^\circ} \theta^\circ = 1$\\
$px + qy \leq 1$\\
\bf{Теорема о дифф}\\
$\cos'_\Omega \theta = -\sin_{\Omega^\circ} \theta^\circ$\\
$\sin_\Omega' \theta = \cos_{\Omega^\circ} \theta^\circ$\\
Давайте напишем полярную замену координат\\
$\begin{cases}
    x = R \cos_\Omega \theta\\
    y = R \sin_\Omega \theta\\
    |J| = R
\end{cases}$\\
Если $(x(t), y(t))$ - некоторая кривая на $\mathbb{R}^2 \backslash (0, 0)$\\
$R' = ?, \theta' = ?$\\
$x' = R' \cos_{\Omega}\theta - R\theta' \sin_{\Omega^\circ} \theta^\circ$\\
$y' = R' \sin_\Omega \theta + R \theta' \cos_{\Omega^\circ} \theta^circ$\\
$R' = x' \cos_{\Omega^\circ}\theta' + y' \sin_{\Omega^\circ} \theta^\circ$\\
$\theta' = \frac{1}{R}(y' \cos_\Omega \theta - x' \sin_{\Omega} \theta)$
\end{document}